\subsection{}

\subsubsection{Proof of Decomposition Correctness (Process 1)}

The following section demonstrates\\

\underline{MDP Policy Decomposition}\\

First, it is true that the optimal policy $\pi^\ast(s)$ for a centralized MDP will arise from maximizing the expected reward.

\begin{equation*}
\pi^\ast(s) = \argmax_{a} \sum_{s^\prime} R(s,a,s^\prime) P(s^\prime |s, a) + \gamma V(s^\prime)
\end{equation*}

It is now shown that a set of MDPs ($M_i$, $M_k$) can be observed to create an optimal policy $\exists f s.t.\ f\left( \pi^\ast_i, \pi^\ast_k \right) \sim \pi^\ast$.

Given $\pi^\ast_i(S_i, A_k)$, $\pi^\ast_k(S_k)$

\begin{equation*}
1.\quad \pi^\ast(s_i,s_k) = \argmax_{a_i, a_k} \sum_{s^\prime_i} \sum_{s^\prime_k} R\left( (s_i,s_k),(a_i,a_k),(s^\prime_i,s^\prime_k)\right) P\left( (s^\prime_i, s^\prime_k) \middle| (s_i, s_k), (a_i, a_k)  \right)
\end{equation*}

\textasteriskcentered \quad \underline{Lemma 1 -- augmentation with $a_k$}\quad where $a^\prime_k=\pi^\ast(s^\prime_k)$

\begin{equation*}
2.\quad\pi^\ast(s_i, s_k) = \argmax_{a_i, a_k} \sum_{s^\prime_i} \sum_{s^\prime_k} R\left((s_i, s_k, a_k), (a_i,a_k),(s^\prime_i,s^\prime_k,a^\prime_k) \right) P\left( (s^\prime_i, s^\prime_k, a^\prime_k) \middle| (s_i, s_k), (a_i,a_k) \right) 
\end{equation*}

\textasteriskcentered \quad \underline{Lemma 2 -- Simplification}\\

$\overbrace{\ast\quad\text{assume} \quad \argmax_{a_i,a_k} \equiv \argmax_{a_i} \argmax_{a_k}}^{\text{separability}}$\\

 \begin{equation*}
3.\quad\pi^\ast(s_i, s_k) = \argmax_{a_i, a_k} \sum_{s^\prime_i}  R\left((s_i, a_k), (a_i,a_k),(s^\prime_i, a^\prime_k) \right) P\left( s^\prime_i, a^\prime_k \middle| (a_i,a_k),  (s_i, s_k) \right) 
\end{equation*}

\underline{Lemma 3}\\

\textasteriskcentered \quad separation of $a_k$, and $a_k \leftarrow \pi^{\ast}_k(s_k)$

 \begin{equation*}
4.\quad\pi^\ast(s_i, s_k) =\left( \argmax_{a_i} \sum_{s^\prime_i}  R\left((s_i, a_k), a_i,(s^\prime_i, a^\prime_k) \right) P\left( s^\prime_i, a^\prime_k \middle| a_i,  (s_i, s_k) \right) \right)
\end{equation*}

\textasteriskcentered \quad \underline{Lemma 4}\\

\begin{equation*}
a_i = \pi^\ast_i(s_i) \longrightarrow^{\cup} \left( \argmax_{a_k} \sum_{s^\prime_k}  R\left(s_k, a_k, s^\prime_k, a^\prime_k \right) P\left( s^\prime_k \middle| a_k, s_k \right) \right)
\end{equation*}

\begin{center}
\fbox{
\begin{minipage}{5cm}
\begin{equation*}
5.\quad\pi^\ast(s_i, s_k) = \pi^\ast_i(s_i,a_k)\cup \pi^\ast(s_k)
\end{equation*}
\end{minipage}
}
\end{center}

Thus, given $\pi^\ast_i$ and $\pi^{i}_{k}$, it is possible to infer $\pi^\ast \left( s_i, s_k \right)$.\\

We now prove lemmas involved.\\

Lemma 1:\\
\vspace*{4cm}

Lemma 2:\\
\vspace*{4cm}

Lemma 3:\\
\vspace*{4cm}

Lemma 4:\\
\vspace*{4cm}