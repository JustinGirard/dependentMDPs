\section{Mapping function}
\label{sec:mapping}

After defining Parent and Child MDPs, it is important to note how to map information from a centralized MDP onto a parent-child pair: (equation $d^{M_j, M_k}_{M_i}$) $d^{M_j, M_k}_{M_i}$ . To expedite convergence and computational proofs that follow, a mapping function $d^{M_j, M_k}_{M_i}$ is assume to have Basic Requirements (Sec.~\ref{sec:basicrequirements}). Afterward the process is explained (Sec.~\ref{sec:mappingprocess}).

\subsection{Basic mapping requirements}
\label{sec:basicrequirements}

$S_i$, $S_j$, $S_k$: \quad $S_j \times (S_k/S_j)  \supseteq S_i$,\quad $S_k \times (S_j/S_k) \supseteq S_i$\\

Sets $S_k$ and $S_j$ must be able to be combined to make the original set.\\

$A_i$, $A_j$, $A_k$: \quad $A_i \subseteq A_j\cup A_k$\\

Also, action sets must combine to recover the original action set.\\

$T_i$, $R_i$ $\sim$ unknown/unknowable, stable decomposition\\

The true transition and reward functions may remain unknown.\\

more$\longrightarrow$
\fbox{\pbox[b]{\textwidth}{\qquad\qquad assumed\\
$\ast$ important to select so that $\tilde{T}_i$ \& $\tilde{T}_k$ seem independent}}\\

\subsection{Mapping process}
\label{sec:mappingprocess}

Transition mapping:\\

$\exists f_1:\tilde{T}_i\to \tilde{T}_j, \tilde{T}_k$,\ invertible; $\tilde{T}_i = f_1\left(f^{-1}\left(\tilde{T}_i\right)\right)$\\
$\exists f_2:\tilde{R}_i\to \tilde{R}_j, \tilde{R}_k$,\ invertible; $\tilde{R}_i = f_2\left(f_2^{-1}\left(\tilde{R}_i\right)\right)$\\

\underline{Transition function mapping: knowing $s_x\in S_x$, $s_y\in S_y$} (1 way)\\

Goal $\exists f:P(S_x|A,S_x)\leftarrow P(S|A,S)$\\
knowing $P(S_x \times S_y | A_x \cup A_y, S_x \times S_y)=P(S|A,S)$

Clearly: 
\begin{equation*}
P(s^\prime_x|s_x,a_x) = \sum_{s_y^\prime}\sum_{a_y}\sum_{s_y} P\left( s_x^\prime,s_y^\prime|s_x,s_y,a_x,a_y\right)P\left(a_y|S_y \right)P\left( s_y \right)
\end{equation*}

\begin{equation*}
\therefore\quad\tilde{T}\left( s^\prime_x \middle| s_x, a_x \right) = \sum_{s^\prime_y} \sum_{a_y} \sum_{s_y} \tilde{T}\left( s^\prime \middle| a, s  \right) \underbrace{\tilde{\pi}\left( a_y \middle| s_y \right)}_{\text{require policy mapping}}P(s_y)
\end{equation*}

Action mapping:\\

\underline{Action mapping} (1 way)\\

Next, we can consider an action mapping where actions from $A$ can be randomly assigned to $A_x$, $A_y$: $A_x \leftarrow \left\{ a\in A^\prime \middle| A^\prime \subseteq A \right \}$, $A_x, A_y\subseteq A$, $A_x \cup A_y = A$, $A_x \neq \{ \null \}$, $A_y \neq	 \{ \null \}$.\\

\underline{General approach}: \underline{High reward for \textbf{???}ve states}\\

Given $\tau \in \mathbb{R}$, $\pi(a|s)$, $\tilde{Q}(a,s)$ then 
\begin{equation*}
A_x \leftarrow A_x \cup \left\{ a  \middle| \underbrace{ \pi(a|s) \tilde{Q}(a,s) > \tau }_{\text{condition}} \right\}
\end{equation*}
or, more usefully/generally
\begin{equation*}
A_x \leftarrow A_x \cup \left\{ a \middle| \underbrace{ \left( \pi (a \middle|  S_s^{R^\prime} \right) \tilde{Q}(a,S^{*\prime}) > \tau }_{\text{condition}} \right\}
\end{equation*}
where
\begin{equation*}
S_s^{*\prime}=\left\{ S^\prime \middle| S/s_s^* \neq S\right\} \qquad\text{(see p. ???)}
\end{equation*}

Condition options:
\begin{equation*}
\ast \text{reformulation over set }S\text{\ vs.\ }s\in S\qquad 
\left\{ 
\begin{array}{l}
\text{a)\ }\pi(a|s)\tilde{Q}(a,s)>\tau \qquad \cdots \quad \text{High reward} \\
\text{b)\ }\pi(a|s)\tilde{Q}(a,s)>\tau,\quad\pi(a|s)>0\qquad \cdots \quad \text{small reward}
\end{array}
\right.
\end{equation*}



(Network approach)\\ 


